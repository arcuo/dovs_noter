\section{LLVM Intermediate Representation}

Using LLVM as a bridge from many high-level languages, to many low-level languages. Makes it so that we only have to make $n+m$ combinations from $n\times m$.\\

\textbf{Low-level aspects}:

\begin{itemize}
    \item Convenient abstraction over architecture specific issues (we do not want to deal with.)
    \begin{itemize}
        \item Infinite registers (we do not care about exact purpose of registers)
        \item Direction of stack growth
        \item Calling convention
        \item Exact instruction set
    \end{itemize}
    \item Exposes low-level aspects important for code gen:
    \begin{itemize}
        \item Notion of functions (call stack)
        \item Allocation of stack during function execution
        \item Storing and loading pointers (stack and heap locations)
        \item Instructions for arithmetics
        \item Instructions for jumps and conditional jumps
    \end{itemize}
\end{itemize}

\subsubsection{LLVM features}:

\begin{itemize}
    \item Intermediate assembly like
    \item Infinite number of registers - can only be assigned to once (single static assignment)
    \item Types
    \begin{itemize}
        \item First-class types*: 
        \begin{itemize}
            \item Single value: \textbf{integer}, float, x86\_mmx, \textbf{pointer}, vector, label
            \item Aggregate types: \textbf{arrays}, \textbf{structures}
        \end{itemize}
        \item Function type and void type
    \end{itemize}
    \item Identifiers
    \begin{itemize}
        \item Global (@)
        \item Local (\%)
        \item LLVM-- only allows named identifiers
    \end{itemize}
    \item 
\end{itemize}

*\textit{Bold are used by us.}\\

\subsubsection{Structure of LLVM}

\begin{itemize}
    \item Program:
    \begin{itemize}
        \item Global decls (list of globals, e.g. string literals)
        \item Types: list of named types
        \item Function decls
    \end{itemize}
    \item Function declaration: Header (param, return type), Control Flow Graph (Entry basic block + labeled basic blocks)
    \item Control Flow Graph:
    \begin{itemize}
        \item Basic block: List of instructions + terminator
        \item Terminator: Return or branch instruction
    \end{itemize}
\end{itemize}
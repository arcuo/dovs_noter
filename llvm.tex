\section{LLVM Intermediate Representation}

Using LLVM as a bridge from many high-level languages, to many low-level languages. Makes it so that we only have to make $n+m$ combinations from $n\times m$.\\

\textbf{Low-level aspects}:

\begin{itemize}
    \item Convenient abstraction over architecture specific issues (we do not want to deal with.)
    \begin{itemize}
        \item Infinite registers (we do not care about exact purpose of registers)
        \item Direction of stack growth
        \item Calling convention
        \item Exact instruction set
    \end{itemize}
    \item Exposes low-level aspects important for code gen:
    \begin{itemize}
        \item Notion of functions (call stack)
        \item Allocation of stack during function execution
        \item Storing and loading pointers (stack and heap locations)
        \item Instructions for arithmetics
        \item Instructions for jumps and conditional jumps
    \end{itemize}
\end{itemize}

\subsubsection{LLVM features}:

\begin{itemize}
    \item Intermediate assembly like
    \item Infinite number of registers - can only be assigned to once (single static assignment)
    \item Types
    \begin{itemize}
        \item First-class types*: 
        \begin{itemize}
            \item Single value: \textbf{integer}, float, x86\_mmx, \textbf{pointer}, vector, label
            \item Aggregate types: \textbf{arrays}, \textbf{structures}
        \end{itemize}
        \item Function type and void type
    \end{itemize}
    \item Identifiers
    \begin{itemize}
        \item Global (@)
        \item Local (\%)
        \item LLVM-- only allows named identifiers
    \end{itemize}
    \item 
\end{itemize}

*\textit{Bold are used by us.}\\

\subsubsection{Structure of LLVM}

\begin{itemize}
    \item Program:
    \begin{itemize}
        \item Global decls (list of globals, e.g. string literals)
        \item Types: list of named types
        \item Function decls
    \end{itemize}
    \item Function declaration: Header (param, return type), Control Flow Graph (Entry basic block + labeled basic blocks)
    \item Control Flow Graph:
    \begin{itemize}
        \item Basic block: List of instructions + terminator
        \item Terminator: Return or branch instruction
    \end{itemize}
\end{itemize}

\textbf{Conversion form Tiger to LLVM}:
\begin{itemize}
    \item\textit{Mutable values}: In LLVM we can only assign to a register once. Instead use \texttt{alloca} to allocate space on the stack and \texttt{store} the pointer. Then use \texttt{load} and \texttt{save} to read and update the variable.
    \item \textit{Mutable values continued}: The number of locals is statically known and at compile time, traverse AST to create identifiers for each variable with pointer.
    \item When compiling a function: Emit a bunch of \texttt{alloca}s for all the locals in the function first.
    \item \textit{Nested functions}: Allocate all locals in record structure. Access locals and parent scope via GEP. Add static link to parent function as first argument in the function and first element in the locals structure.
    \item \textit{Hoisting}: All functions are at the same level. Add variable/identifier offset (depth) to AST so that we can use it for GEP.
    \item \textit{Conditionals}: Reserve return location and then store Then/Else bodies return there.
    \item \textit{Allocation}: \texttt{alloca}s need to happen in the first basic block, so that they are not repeated, for example during a loop, growing the stack unnecessarily.
    \item \textit{Records}: Nested records are flattened into \texttt{i8*} (pointer) and accessed via GEP. Records are allocated on the heap. The size is calculated via runtime C (\texttt{calloc}).
    \item \textit{Arrays}: Same as for records (allocation, size and static links).
\end{itemize}

\subsection{Closure conversion}

In functional programming a function passed as an argument is represented as a closure: Pointer and means of accessing non-local variables (free variables).

\begin{lstlisting}
    F = {fun*, env*}
\end{lstlisting}    

All functions are top-level and closed.\\

Closure conversion phase in functional compiling transforms the functions so non of them appear to access free variables, turning all free variable access into formal parameter access.

\begin{align*}
    &f(a_1, a_2, \dots, a_n) = B \text{ at depth }d\\
    &x_1, x_2, \dots, x_n \text{ are escaping local variables,}\\
    &y_1, y_2, \dots, y_n \text{ are non-escaping local variables,}\\
    &\text{rewrite into:}\\
    & f(a_0, a_1, \dots, a_n) = \texttt{let var }r :=\{a_0, x_1, x_2, \dots, x_n\} \texttt{ in } B' \texttt{ end}
\end{align*}

\begin{itemize}
    \item New parameter $a_0$ is the static link.
    \item Variable $r$ is a record containing all escaping variables and the static link.
    \item The variable $r$ becomes the static link argument when calling a function at depth $d+1$.
    \item In LLVM, use the GEP function to access free variables in some offset of $r$.
\end{itemize}

These are called \textit{linked closures}. In immutable variable languages, just copy all closed values into closure instead of static link (\textit{flat closures}). \textit{Hybrid closures} are the best of both worlds.

\subsubsection{First-class (Higher-order) functions}

You can call a function, that has two arguments, with only one argument. It then return a function call variable, that you can call with the last argument. Closure conversion also hoists first-class functions into top-level and closed functions.

